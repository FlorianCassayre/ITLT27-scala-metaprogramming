\begin{frame}{Macros: quotation and splicing}

  \begin{itemize}
    \item \textbf{Quotation}: converting \textbf{code} into an \textbf{AST}
    \item \textbf{Splicing}: converting an \textbf{AST} back into \textbf{code}
  \end{itemize}

\begin{center}
\begin{tikzpicture}[x=1cm, y=1cm]
    \useasboundingbox (-6,-2) rectangle (6,2);

    \node (quote) at (-3,0) {\huge\texttt{T}};
    \node (splice) at (3,0) {\huge\texttt{Expr[T]}};
    
    \draw[->, very thick]
        ([xshift=10pt, yshift=10pt]quote.east) -- ([xshift=-10pt, yshift=10pt]splice.west) 
        node[midway, above, yshift=10pt] {\Large\texttt{'\{ * \}}};
    
    \draw[<-, very thick]
        ([xshift=10pt, yshift=-10pt]quote.east) -- ([xshift=-10pt, yshift=-10pt]splice.west) 
        node[midway, below, yshift=-10pt] {\Large\texttt{\$\{ * \}}};
\end{tikzpicture}
\end{center}

  \note{
    \begin{itemize}
      \item So far, we have seen how to inline code. Macros build on top of this construct and allow transforming code at compile-time
      \item On the left we have the representation of a value; on the right is the representation of a piece of code that when executed produces this value
      \item The two arrows correspond to the two operations that enable converting one to the other; quotation (at the top) and splicing (at the bottom). They are dual from one another
    \end{itemize}
  }

\end{frame}
