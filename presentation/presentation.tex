\documentclass[aspectratio=169]{beamer}

\usepackage{hyperref}
\usepackage{tikz}
\usetikzlibrary{calc}
\usetikzlibrary{positioning}

% SVG

\usepackage{svg}
\svgpath{{images/}}

% Code blocks

\usepackage{listings}
\usepackage{xcolor}
\usepackage{tcolorbox}
\tcbuselibrary{listings, skins}

% Mimicked style from https://docs.scala-lang.org/
% (with some changes, since listings is quite limited)
\definecolor{scala-background}{RGB}{253,253,247}
\definecolor{scala-border}{RGB}{229,234,234}
\definecolor{scala-keyword}{RGB}{153,0,0}
\definecolor{scala-identifier}{RGB}{47,138,210}
\definecolor{scala-comment}{RGB}{153,153,136}
\definecolor{scala-string}{RGB}{218,50,47}

\lstdefinestyle{scalaStyle}{
    language=Scala,
    keywordstyle=\color{scala-keyword}\bfseries,
    commentstyle=\color{scala-comment}\itshape,
    stringstyle=\color{scala-string},
    identifierstyle=\color{scala-identifier},
    numbers=none,
    tabsize=2,
    showspaces=false,
    showstringspaces=false,
    breaklines=true,
    breakatwhitespace=true,
    backgroundcolor=\color{scala-background},
    frame=single,
    rulecolor=\color{scala-background},
    belowskip=0pt,
}

\newtcblisting{scalacode}[1][\small]{
    listing only,
    listing style=scalaStyle,
    listing options={
        style=scalaStyle,
        basicstyle=\ttfamily#1,
    },
    colback=scala-background,
    colframe=scala-border,
    boxrule=0.5pt,
    arc=0.5mm,
    left=0.5pt,
    right=0.5pt,
    top=0.5pt,
    bottom=0.5pt,
}

% Theme

\setbeamertemplate{navigation symbols}{} % Hide navigation

\usetheme{default}

\definecolor{color-logo}{RGB}{222,52,35}

\setbeamercolor{title}{fg=color-logo}
\setbeamercolor{subtitle}{fg=black}

\setbeamercolor{frametitle}{fg=white,bg=color-logo}

\setbeamercolor{item}{fg=color-logo}

% Other

\newcommand{\compilesto}{\centering{\Large{$\Downarrow$}}}

\title{Metaprogramming in Scala}
\subtitle{CERN lightnight talk}
\author{Florian Cassayre}
\date{7th November 2025}

\begin{document}

% Notes
%\setbeameroption{show notes}
\setbeameroption{hide notes}

\maketitle

\note{
  \begin{itemize}
      \item In this short presentation, I will give an introduction about metaprogramming
      \item I will illustrate some of the technique using the Scala programming language
      \item No prior experience with Scala is expected
  \end{itemize}
}


\section{Introduction}
\subsection{Metaprogramming}
\begin{frame}{Metaprogramming}

  \begin{block}{}
    A set of techniques in which programs have the ability to
    \textit{analyze}, \textit{generate}, or \textit{transform} other programs---including themselves.
  \end{block}

  \begin{block}{}
    Goals:
  \end{block}
  \begin{itemize}
    \item Increase abstraction
    \item Reduce boilerplate
    \item Improve runtime performance
  \end{itemize}

  \begin{block}{}
    Some examples (of varying capabilities):
  \end{block}

  \begin{itemize}
    \item \textbf{C++}: templates
    \item \textbf{JavaScript}: \texttt{eval}, proxy objects, reflection
    \item \textbf{Java}: reflection, annotation processors
  \end{itemize}

  \note{
    \begin{itemize}
        \item To put it shortly, metaprogramming is a set of techniques where programs can treat other programs as their data
        \item There are several motivations for this, such as to increase abstraction and reduce boilerplate by generating code, improve the runtime performance by optimizing the source code, introspection, and so on
        \item Find below how those techniques appear in these languages
    \end{itemize}
  }

\end{frame}

\subsection{Scala}
\begin{frame}{Metaprogramming in Scala 3}

  \begin{itemize}
    \item \textbf{Runtime reflection}: access the high level structure
    \item \textbf{Inlining}: a modifier to guarantee inline at point of use
    \item \textbf{Macros}: convert program code to data, and vice versa
    \item \textbf{Runtime multi-staging}: compile code at runtime
    \item \textbf{Compile-time operations}: e.g. convert types to values
  \end{itemize}

  \note{
    \begin{itemize}
        \item The metaprogramming features of Scala 3 are as follows:
        \item A reflection system; most programming languages exhibit some sort of a reflection API. For Scala, because it runs on the JVM it inherits the same API. It is however more powerful thanks to TASTy trees
        \item Scala defines a soft keyword "inline" to replace a name by its expression. This feature is generally not very expensive to implement within an existing compiler. Alone, they enable in certain cases to compile code in a more efficient manner
        \item The main use case for inlining remains macros, which build on top of it. Macro enable transforming code at compilation-time, in a type-safe way
        \item Runtime multi-stage programming is a generalization of macros where the compilation can happen at runtime
        \item Finally, compile-time operations are helper tools that can for instance allow use to materialize types into concrete values, or raise compilation errors programmatically
    \end{itemize}
  }

\end{frame}

\begin{frame}{Scala}
  \begin{itemize}
    \item Invented in 2004 at EPFL
    \item Runs on the JVM
    \item Statically typed, multi-paradigm and versatile
    \item Focus on type safety and expressiveness
    \item Complete revamp of the language in version 3
  \end{itemize}

  \vfill

  \begin{center}
    \includesvg[width=0.4\textwidth]{scala-logo}
  \end{center}

  \note{
    \begin{itemize}
        \item Scala was invented 20 years ago at EPFL, in Lausanne, while trying to address some of the limitations of Java. It runs on the JVM
        \item It is statically typed with inference, and is designed to be a general purpose language
        \item It has a powerful type system compared to other programming languages, which happens to also be Turing-complete as we will see later
        \item To be noted that the language has been recently rewritten from ground-up, enabling some of the metaprogramming features natively
    \end{itemize}
  }
  
\end{frame}

%\begin{frame}[fragile]{Scala}

  \begin{block}{}
    A simple Scala program:
  \end{block}

\begin{scalacode}
def add(a: Int, b: Int): Int =
  a + b

val c = add(1, 2)

println(c) // 3
\end{scalacode}

\end{frame}

\section{Inlining and compile-time operations}
\subsection{Inlining}
%\begin{frame}[fragile]{Function inlining}

  \begin{block}{}
    Replace a function \textbf{call site} with the \textbf{body} of the called function.
  \end{block}

  \vspace{0.25cm}

  \pause

\begin{scalacode}
inline def add(a: Int, b: Int): Int = a + b
\end{scalacode}

  \pause

\vfill

\begin{scalacode}
val c = add(1, 2)
\end{scalacode}

  \pause

  \compilesto

\begin{scalacode}
// 1 + 2
val c = 3
\end{scalacode}

  \vspace{0.5cm}

\end{frame}

\begin{frame}[fragile]{Function inlining: recursion}

  \begin{block}{}
    Replace a function \textbf{call site} with the \textbf{body} of the called function.
  \end{block}

\begin{scalacode}
inline def factorial(n: Int): Int =
  if n == 0 then 1
  else n * factorial(n - 1)
\end{scalacode}

  \vfill

\begin{scalacode}
val f = factorial(3)
\end{scalacode}

  \pause

  \compilesto

\begin{scalacode}
// 3 * (2 * (1 * (1))
val f = 6
\end{scalacode}

(the compiler also performs a \textbf{constant folding} optimization)

  \note<1>{
    \begin{itemize}
        \item Inlining also works with recursive functions
        \item Here is a classic example with the factorial function
    \end{itemize}
  }

  \note<2>{
    \begin{itemize}
        \item Again, thanks to constant folding we ultimately this code compiles to a literal, essentially "computing" the result at compilation-time
        \item In fact, inlining alone is Turing-complete. This comes to no surprise as lambda calculus is itself Turing-complete
    \end{itemize}
  }

\end{frame}

%\begin{frame}[fragile]{Function inlining: transparency}

  \begin{block}{}
    Transparent inline enables precise type-checking:
  \end{block}

\begin{scalacode}
transparent inline def add(a: Int, b: Int): Int = a + b
\end{scalacode}

\begin{scalacode}
val c: 3 = add(1, 2)
\end{scalacode}

  \vfill

  \pause

  \begin{itemize}
      \item Scala offers literal-based singleton types
      \item Supports constant folding at the type level
      \item Two types coerce if one can be converted to the other using \textbf{reduction}, \textbf{subtyping} or \textbf{implicit conversions}
  \end{itemize}

  \note<1>{
    \begin{itemize}
        \item So far we have seen that the compiler is capable of simplifying expressions thanks to inlining
        \item What about the type? Actually, the types remain the same; there is obviously a conservation property, but the resulting type could be refined based on the final expression after inlining
        \item For this purpose, Scala provides another keyword "transparent", which does exactly that
        \item By adding it to our previous method, we are able to accurately type the result
    \end{itemize}
  }

  \note<2>{
    \begin{itemize}
        \item This example showcases the use of literal types
        \item Notice that the type itself is also constant-folded, this is yet another mechanism of the type system
        \item The example is a bit far-fetched, but in practice this type refinement mechanism can be very effective when dealing with branches or pattern matching
    \end{itemize}
  }

\end{frame}
%\subsection{Type operations}
%\begin{frame}[fragile]{Type operations}
\begin{scalacode}
type Gcd[A <: Int, B <: Int] =
  B match
    case 0 => A
    case _ => Gcd[B, A % B]
\end{scalacode}

\begin{scalacode}
val gcd = constValue[Gcd[48, 18]]
\end{scalacode}

\compilesto

\begin{scalacode}
val gcd = 6
\end{scalacode}

\end{frame}
\section{Macros}
\subsection{General principle}
\begin{frame}{Scala compiler}

  \begin{block}{}
    A compiler takes \textbf{code} and produces \textbf{machine code} (or \textbf{bytecode} in the context of the JVM).
  \end{block}

  \vfill

\begin{center}
\scalebox{1}{%
\begin{tikzpicture}[
  node distance=1cm,
  jvm/.style={
    draw,
    rectangle,
    rounded corners=2pt,
    minimum width=2cm,
    minimum height=2cm,
    align=center
  }
]

  \node<1->[jvm, thick] (jvm) at (0,0) {\only<1>{Compiler}\only<2->{Scala\\Compiler}};

  \draw<1->[->, thick] (-4,0) -- (jvm) node[midway, fill=white, inner sep=1pt] {Code};

  \draw<1->[->, thick] (jvm) -- (4,0) node[midway, fill=white, inner sep=1pt] {Bytecode};

  \draw<2->[->, thick] (0.5,-1) -- (0.5,-1.75) -- (-0.5,-1.75) -- (-0.5,-1);
  \node<2-> at (0,-2.1) {Bytecode};
  
\end{tikzpicture}
}
\end{center}

  \vfill

  \pause

  \begin{block}{}
    When the Scala compiler encounters a macro, it \textbf{compiles} it, \textbf{executes} the compiled bytecode, and then proceeds to compiles the returned \textbf{AST}.
  \end{block}

  \note<1>{
      \begin{itemize}
          \item Before moving on to the macros, this slide naively describes the behavior of a standard compiler
          \item Essentially taking code as input and producing machine code as the output
      \end{itemize}
  }

  \note<2>{
      \begin{itemize}
          \item Most compilers are actually written in the language that they compile, and more importantly their machine code is typically the same as what they produce
          \item In that case, why not allow the compiler to execute the code it has just compiled?
          \item This is precisely what macro compilers do, in particular the Scala compiler
          \item Macros are compiled to bytecode, this bytecode is then executed to produce an AST that will be used throughout the compilation of the remaining code 
      \end{itemize}
  }
\end{frame}
\begin{frame}{Macros: quotation and splicing}

  \begin{itemize}
    \item \textbf{Quotation}: converting \textbf{code} into an \textbf{AST}
    \item \textbf{Splicing}: converting an \textbf{AST} back into \textbf{code}
  \end{itemize}

\begin{center}
\begin{tikzpicture}[x=1cm, y=1cm]
    \useasboundingbox (-6,-2) rectangle (6,2);

    \node (quote) at (-3,0) {\huge\texttt{T}};
    \node (splice) at (3,0) {\huge\texttt{Expr[T]}};
    
    \draw[->, very thick]
        ([xshift=10pt, yshift=10pt]quote.east) -- ([xshift=-10pt, yshift=10pt]splice.west) 
        node[midway, above, yshift=10pt] {\Large\texttt{'\{ * \}}};
    
    \draw[<-, very thick]
        ([xshift=10pt, yshift=-10pt]quote.east) -- ([xshift=-10pt, yshift=-10pt]splice.west) 
        node[midway, below, yshift=-10pt] {\Large\texttt{\$\{ * \}}};
\end{tikzpicture}
\end{center}

  \note{
    \begin{itemize}
      \item So far, we have seen how to inline code. Macros build on top of this construct and allow transforming code at compile-time
      \item On the left we have the representation of a value; on the right is the representation of a piece of code that when executed produces this value
      \item The two arrows correspond to the two operations that enable converting one to the other; quotation (at the top) and splicing (at the bottom). They are dual from one another
    \end{itemize}
  }

\end{frame}

\subsection{Examples}
\begin{frame}[fragile]{Assertion macro}

\begin{scalacode}[\footnotesize]
inline def assert(inline cond: Boolean): Unit =
  ${ assertImpl('cond) }

def assertImpl(cond: Expr[Boolean])(using Quotes): Expr[Unit] =
  '{ if !$cond then throw AssertionError(${Expr(cond.show)}) }
\end{scalacode}

    \pause

\begin{scalacode}
assert("foo".equals("bar"))
\end{scalacode}

    \pause

    \compilesto

\begin{scalacode}
if !"foo".equals("bar") then
  throw AssertionError("\"foo\".equals(\"bar\")")
\end{scalacode}

  \note<1>{
    \tiny
    \begin{itemize}
      \item This first example demonstrates how to implement an asserting function
      \item It takes a condition parameter, checks that it holds and otherwise raises an exception with a developer-friendly message
      \item In this case we are interested in printing the expression passed as argument, reason why we would need to use macros for that
    \end{itemize}
  }

\end{frame}
%\begin{frame}[fragile]{Profiling macro}

\begin{scalacode}
inline def timed[R](inline expr: R): R =
  ${ timedImpl('expr) }

def timedImpl[R: Type]
  (expr: Expr[R])(using Quotes): Expr[R] = '{
  val t0 = System.nanoTime()
  val result = $expr
  val t1 = System.nanoTime()
  val code = ${Expr(expr.show)}
  println(f"$code: ${(t1 - t0) / 1e9}%.6f ms")
  result
}
\end{scalacode}

\begin{scalacode}
// java.lang.Thread.sleep(1000L): 1.000139 ms
timed:
  Thread.sleep(1000L)
\end{scalacode}

  \note{
    \tiny
    \begin{itemize}
      \item In this first macro example, we demonstrate how to implement a profiler. You can see the result of executing it at the bottom of the slide
      \item Basically we will define a \texttt{timed} method, which works as the identity function but will additionally print the time it takes to execute the body along with the body itself
      \item In order to use macros, you need to use the \texttt{inline} keyword on the definition. The parameter inlining is not necessary for macros but in this case will allow us to print the content of the body, rather that some intermediate variable
      \item The implementation is straightforward, we measure the start and end time between the execution of the body, print the difference and finally return the result
      \item Notice that the implementation starts with a quotation, which means that all the code that follows will be generated at every call of \texttt{timed}
      \item Since \texttt{expr} is a variable from the upper stage, we need to splice it; this means that the code it contains will be injected
      \item Finally, we would like to print the expression. For that, Scala offers a method \texttt{show}, which converts the AST into a string that closely matches the source code
      \item Because \texttt{show} returns a \texttt{String}, we have to wrap it into an \texttt{Expr} in order to splice it. Scala offers out of the of the box converters for all primitive values (among other things), but you can build your own
    \end{itemize}
  }

\end{frame}
%\begin{frame}[fragile]{Profiling macro (source)}

\begin{scalacode}[\scriptsize]
inline def timed[R](inline expr: R): R = ${ timedImpl('expr) }

def timedImpl[R: Type](expr: Expr[R])(using Quotes): Expr[R] =
  '{
    val t0 = System.nanoTime()
    val result = $expr // <- Injecting the source code
    val t1 = System.nanoTime()
    val code = ${Expr(expr.show)}
    println(f"$code: ${(t1 - t0) / 1e9}%.6f ms")
    result
  }
\end{scalacode}

\begin{scalacode}[\scriptsize]
val t0 = System.nanoTime()
val result = Thread.sleep(1000L)
val t1 = System.nanoTime()
val code = ${Expr(expr.show)}
println(f"java.lang.Thread.sleep(1000L): ${(t1 - t0) / 1e9}%.6f ms")
result
\end{scalacode}

\end{frame}
\begin{frame}[fragile]{Parsing string contexts}

  \begin{block}{}
    String contexts are syntactic sugars that support splicing (= injection of parameters):
  \end{block}

\begin{scalacode}[\footnotesize]
extension (inline sc: StringContext)
  inline def url(inline args: Any*): URL =
    ${ urlImpl('sc, 'args) }

def urlImpl(scExpr: Expr[StringContext], argsExpr: Expr[Seq[Any]])(using Quotes): Expr[URL] =
  val rawUrl = scExpr.valueOrAbort.parts.mkString
  Try(URL(rawUrl)) match
    case Success(_) => '{ URL(${ Expr(rawUrl) }) }
    case Failure(_) => quotes.reflect.report
      .errorAndAbort(s"Invalid URL: '$rawUrl'")
\end{scalacode}

\begin{scalacode}[\footnotesize]
url"http://example.com"
url"not_a_url" // Compilation error
\end{scalacode}

\end{frame}
\begin{frame}[fragile]{Memoization macro}

\begin{scalacode}[\scriptsize]
@memo
def fibonacci(n: Int): Int =
  if n < 2 then n else fibonacci(n - 2) + fibonacci(n - 1)
\end{scalacode}

\compilesto

\begin{scalacode}[\scriptsize]
private val fibonacci: mutable.HashMap[Tuple1[Int], Int] =
  mutable.HashMap()

@memo
def fibonacci(n: Int): Int =
  fibonacci.getOrElseUpdate(
    Tuple1(n),
    if n < 2 then n else fibonacci(n - 2) + fibonacci(n - 1)
  )
\end{scalacode}

\end{frame}
%\begin{frame}[fragile]{Memoization macro (source)}

\begin{scalacode}[\tiny]
class memo extends MacroAnnotation:
  override def transform(using Quotes)
  (definition: Definition, companion: Option[Definition]): List[Definition] =
    definition match
      case DefDef(name, List(TermParamClause(params)), tpt, Some(body)) =>
        val keyExpr = Expr.ofTupleFromSeq(params.map(p => Ref(p.symbol).asExpr))
        (keyExpr, body.asExpr) match
          case ('{ $key: keyType }, '{ $body: bodyType }) =>
            val cacheType = TypeRepr.of[mutable.HashMap[keyType, bodyType]]
            val cacheSymbol = Symbol.newVal(
              definition.symbol.owner, name, cacheType, Private, noSymbol
            )
            val cacheVal = ValDef(
              cacheSymbol,
              Some('{ mutable.HashMap[keyType, bodyType]() }.asTerm)
            )
            val newBody = '{
              ${ Ref(cacheSymbol).asExprOf[mutable.HashMap[keyType, bodyType]] }
                .getOrElseUpdate($key, $body)
            }.asTerm
            val newDef = DefDef.copy(definition)(
              name, List(TermParamClause(params)), tpt, Some(newBody)
            )
            List(cacheVal, newDef)
      case _ => report.errorAndAbort("@memo cannot be used in this context")
\end{scalacode}

\end{frame}
\begin{frame}[fragile]{Runtime multi-stage programming}

  \begin{block}{}
    The runtime can be bundled with the Scala compiler itself. This can be used to implement a typesafe JIT compiler:
  \end{block}

  \vfill

\begin{scalacode}
given Compiler = Compiler.make(getClass.getClassLoader)

staging.run:
  '{ println("Hello world!") }
\end{scalacode}

\end{frame}

\section{Libraries}
\begin{frame}{Scala libraries}

  \begin{itemize}
      \item \href{https://github.com/fthomas/refined}{\textbf{refined}}: data constraints represented as types
      \item \href{https://github.com/typelevel/shapeless-3}{\textbf{shapeless}}: a framework that implements \href{https://en.wikipedia.org/wiki/Generic_programming}{\textit{generic programming}}
      \item \href{https://github.com/zio/zio-quill}{\textbf{quill}}: DSL for type-safe SQL queries
  \end{itemize}

\end{frame}
\section{Conclusion}
\begin{frame}

  \begin{block}{}
    \centering{\small\href{https://github.com/FlorianCassayre/ITLT27-scala-metaprogramming}{FlorianCassayre/ITLT27-scala-metaprogramming}}
  \end{block}

\end{frame}


\end{document}